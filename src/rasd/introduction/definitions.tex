\section{Definitions, acronyms, abbreviations}
\begin{description}
    \item[Non registered user:] He/she is a common person who is not registered into the system and want to register.
	\item [Logged User:] He/she is the person who has previously registered to the service and use its functionalities . In order to use he/she has to be logged into the system;
	\item[Banned User:] He/she is a user that has been banned from the system;
	\item[Passengers:] They are common people that don't need to be registered into the system;
	\item[Maintainer:] He/she is an employee that is responsible to recharge and move the cars;
	\item [Car:] The object the user can reserve. It is uniquely identified by its plate. It could be in one of three different states:
	\begin{itemize}
		\item Available: the car can be reserved.
		\item Booked: the car is reserved but who did it has not yet arrived.
		\item Busy: the car is reserved and who did it is in the car itself.
		\item Unavailable: the car cannot be reserved because it has some problems.
	\end{itemize}
	Each car has its own capacity and battery level.
	\item [Safe Area:] An area where the user can park the car in order to end the rental. It is uniquely identified by a GPS coordinate and a range.
	\item[Special Area:] A safe are featured with a power grid station.
	\item[Bill:] It is the amount of money that the user has to pay at the end of the ride;
	\item[Tablet:] It is the tablet that is present in every car;
	\item [Power Grid Station:] A spot where the user can park the car and charge it, in order to end the rental and obtain the relative discount. It is uniquely identified by a GPS coordinate.
	\item [On-Board Computer:] The object that communicate with server, in order to update car status and calculate rental fee.
	
\end{description}
