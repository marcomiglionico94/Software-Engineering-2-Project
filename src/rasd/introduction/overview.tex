\section{Overview}
This document is the Requirement Analysis and Specification Document for the PowerEnJoy system.
Its main task is to give a specification of the requirements that our system has to fulfil, using both natural language and models.
It explains both the application domain and the system to be developed and it constitutes a baseline for the project planning and the estimation of size, cost and schedule.

Furthermore, it contains the description of the scenarios, the use cases that describe them, the constraints of our system, and the relationships with the external world. Finally it also formalize the specification of some features of the applications, using Alloy language.

The remainder of this document is divided into two chapters:
\begin{description}
	\item [\autoref{ch:overall_description}. Overall Description:] This chapter contains the product perspective, his functions and constraints, some mockups of the mobile application, the actors and the assumptions. It also contains the most significant scenarios of our system.
	This chapter has been written in order to be read from both the customer and the users, therefore it does not contain technical terms.
	\item [\autoref{ch:specific_requirements}. Specific Requirements:] This chapter goes into the detail of functional and non-functional requirements, the alloy code and the user characteristics.
	The target of this chapter are the team members (project managers, developers, testers and so on), so it is not written in a way to be understood by the customer or the users.
\end{description}
