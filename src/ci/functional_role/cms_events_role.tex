\section{Cms Events}

The Cms Events class helps to manage the incoming http requests and to properly set the equivalent response.

In the public wiki, this class is mentioned in the page "OFBiz Content Management How to"\footnote{https://cwiki.apache.org/confluence/display/OFBIZ/OFBiz+Content+Management+How+to}. Here, you can understand that this class is useful if you want to set-up a content driven website. In fact, to do a quickly initial set-up, you can add to the controller.xml file the following (or similar) entries:

\lstset{
    language=xml,
    tabsize=3,
    %frame=lines,
    caption=Test,
    label=code:sample,
    frame=shadowbox,
    rulesepcolor=\color{gray},
    xleftmargin=20pt,
    framexleftmargin=15pt,
    keywordstyle=\color{blue}\bf,
    commentstyle=\color{OliveGreen},
    stringstyle=\color{red},
    numbers=left,
    numberstyle=\tiny,
    numbersep=5pt,
    breaklines=true,
    showstringspaces=false,
    basicstyle=\footnotesize,
    emph={path},emphstyle={\color{magenta}}}
    \lstinputlisting{functional_role/cms_events_example.xml}
    
In this way, by default all the incoming requests will be dispatched to the CmsEvents event.
This event will use the data in the Content data model to generate the content of the page and will return it back to the browser.
In this way it will be possible to add new pages just editing the data in the Content data model and without editing the controller.xml file.

In the following we will see more details about each methods of the class. 

\subsection{cms method}

Given an http request and an http response, the cms method checks if the request is valid or not and it assigns the proper status code to the http response.
The status code is provided by the verifyContentToWebSite function, that can return three status codes: "OK", "NOT FOUND" and "GONE".
If the status code is not "OK", then the function cms tries to find a specific error page for this website, concerning the status code.
If this is not possible, then it tries to find a generic (not related to this current website) content error page concerning the status code.

Furthermore, if the status code found with the verifyContentToWebSite function was "OK" or we found an error page with one of the two previous researches, then we call the function ContentWorker.renderContentAsText.
One of the parameter of this function is "content id". If the status code of the response is "OK", then the content id passed to the function will refer to the one specified in the request, otherwise it will refer to the error page that was found.

%The cms method starts taking some info from the http servlet request that was passed to the function, i.e. the delegator, the local dispatcher, the servlet context and the http session, the locale, the website id, the target request and the actual request. It also checks the validity of these last three variables and it manages the case in which they are not specified in the request.

%If in the session it is indicated to display the maintenance page, then the maintenance page will be rendered as text/html.

%Else, the function identifies the path info of the request, evaluating if there is a specified one or if the default one must be considered and then it parses the path info and gets its alias.
%If the content id of the alias is null, then the request dispatcher forwards the request to the servlet of the request, that has its path elements and parameters adjusted to match the path of the target resource. In this way, the cms method does preliminary processing of the request and the other servlet generates the response. 
%Then it verifies that the request content is associated with the current website and sets the proper status code ("404 NOT FOUND" if the content id is null). If the status code is different from the "200 OK", then the method tries to find a specific Error page for this website concerning the status code

\subsection{verifyContentToWebSite method}
This method takes as input a delegator (used for queries), a Web Site ID and a Content ID.
If the Content ID refers to an actual content of the website referred by the website id the function returns "OK".
If not it checks if that content is a sub content of any publish point of the website, calling the third method of this class.

If the result is positive it returns "OK". Otherwise the function returns "GONE", if the content ID refers to a content no more active, otherwise returns "NOT FOUND".

\subsection{verifySubContent method}
This method takes as input a delegator (used for queries), and two content ID.
It verifies if the first content is an actual sub content of the second or if it belongs to the tree with the second content as a root.

It returns "OK" if that is true, "GONE" if the content is no more active and "NOT FOUND" if there is no connection between the two contents.