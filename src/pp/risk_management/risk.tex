A project development is not without any risk. In this section we will discuss some risks that our project may incur, how likely are they, what are their effects and how we would fix them. We can divide the risks into two main categories: development risk and organizational ones.
\section{Development risks}
\begin{description}
\item[Delays on the deadlines] This is the most common kind of risk during a project. However a well-defined work organization and a good parallelization of the tasks can minimize this risk. In the remote case of a delay on the final release, it may be useful to release a preliminary version of the system, with all the basic functionalities, and adding the advanced ones later.
\item[Lack of communication between members] Our group members can work from home. In this way, they can maximize the efficiency and work in an easier and more comfortable place. However the communication during the development may be more difficult but a rigorous tasks division and a well-written Project Plan may avoid some misunderstandings between members.
\item[Requirement change] Some requirements can change during development. This risk is very difficult to prevent, but writing a more flexible and modular code may reduce the effects.
\item[Code loss] Losing part or all the code is a potentially catastrophic risk. A careful back-up management may avoid this risk.
\item[Integration test failure] One or more failure during the integration test phase could create some delays one the deadlines. To minimize the effect of this risk it is needed to start the integration test phase as soon as possible.
\item[OS update incompatibility] Our application will be used on Android and iOS smartphone. In order to avoid that an update of the OS creates a conflict with our app it is needed to follow all the news made by Google and Apple for developer about further update.
\end{description}

\section{Organizational risks}
\begin{description}
\item[Car issues] Car stock is a critical resource for our system. In order to guarantee an adequate number of cars it is important to schedule a periodic maintenance.
\item[Car theft] The theft of one or more of our cars is a risk not to underestimate. Each car must be equipped with a antitheft system and a GPS to locate eventual lost cars. It is also important to insure each car.
\item[Data loss and data leak] Our system will manage some pieces of personal information, such as the payment ones. We have to protect these data to avoid a loss or a leak. Multiple back-up and adopting industry security standard can minimize this risk. 
\item[Dependency from external services] Our project will use third party services such as Google Maps or the HandyCar system. A change in the terms and conditions of one of these services may reflect on the usability of our system. A general solution for this risk does not exist, but it is our duty to contact third party to negotiate new arrangements. 
\item[Competitivity] Our service is not the first rent service to start. In order to be profitable the competitiveness of our product must be continuously enhanced by introducing innovative features and by keeping our prices competitive.
\end{description}