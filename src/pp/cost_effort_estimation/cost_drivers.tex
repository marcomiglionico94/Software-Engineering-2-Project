\subsection{Cost Drivers}
Cost Drivers are the parameters of the Effort Equation that reflect some characteristics of the developing process and act as multiplicators on the effort needed to build the project. They appear as factors in the Effort Equation. Cost Drivers are described in detail in the COCOMO Manual [6, p. 25].
The cost drivers are the following:
\begin{itemize}
\item \textbf{Required Software Reliability}: Since this is not the only car sharing system in the city,an eventual malfunctioning could led to moderate financial losses,for this reason we set the RELY cost driver to nominal. 

\begin{longtable}{| m{1,8 cm}| m{1,8 cm} | m{1,8 cm} | m{1,8 cm} | m{1,8 cm} | m{1,8 cm} | m{1,8 cm}| }
\hline
\multicolumn{7}{c}{RELY Cost Drivers}\\
\hline
\hline
RELY Descriptors & slightly inconvenience & easy recoverable losses & moderate recoverable losses & high financial losses & risk to human life & \\
\hline
Rating level & very low & low & nominal & high & very high & extra high \\
\hline
Effort multipliers & 0.82 & 0.92 & 1.00 & 1.10 & 1.26 & n/a \\
\hline
\end{longtable}

\item \textbf{Database size}: This measure considers the effective size of our database. We don't have the exact size, but our estimation given the tables and fields we have is to reach a 1GB database. Since it is
distributed over 8.000-12.000 SLOC, the ratio D/P (measured as testing DB bytes/program SLOC) is between 80 and 120 resulting in the DATA cost driver being high.

\begin{longtable}{| m{1,5 cm}| m{1,5 cm} | m{1,5 cm} | m{2,4 cm} | m{2,6 cm} | m{1,5 cm} | m{1,5 cm}| }
\hline
\multicolumn{7}{c}{DATA Cost Drivers}\\
\hline
\hline
DATA Descriptors &  & \begin{equation*}
 {D \over P} {<10} 
\end{equation*} & 
\begin{equation*}
{10\le} {D \over P} {\le 100}
\end{equation*}& 
\begin{equation*}
{100\le} {D \over P} {\le 1000}
\end{equation*} & 
\begin{equation*}
 {D \over P} {>1000} 
\end{equation*}
& \\
\hline
Rating level & very low & low & nominal & high & very high & extra high \\
\hline
Effort multipliers & 0.82 & 0.92 & 1.00 & 1.10 & 1.26 & n/a \\
\hline
\end{longtable}

\item \textbf{Product Complexity}:
Set to high according to the COCOMO II rating scale.

\begin{longtable}{| m{1,8 cm}| m{1,8 cm} | m{1,8 cm} | m{1,8 cm} | m{1,8 cm} | m{1,8 cm} | m{1,8 cm}| }
\hline
\multicolumn{7}{c}{CPLX Cost Drivers}\\
\hline
\hline
Rating level & very low & low & nominal & high & very high & extra high \\
\hline
Effort multipliers & 0.73 & 0.87 & 1.00 & 1.17 & 1.34 & 1.74 \\
\hline
\end{longtable}

\item \textbf{Required Re-usability}:
In our case, the re-usability requirements are just across projects, so the RUSE cost driver is set to nominal.

\begin{longtable}{| m{1,8 cm}| m{1,8 cm} | m{1,8 cm} | m{1,8 cm} | m{1,8 cm} | m{1,8 cm} | m{1,8 cm}| }
\hline
\multicolumn{7}{c}{RUSE Cost Drivers}\\
\hline
\hline
RUSE Descriptors &  & None & Across projects & Across programs & Across product line & Across multiple product line\\
\hline
Rating level & very low & low & nominal & high & very high & extra high \\
\hline
Effort multipliers & n/a & 0.95 & 1.00 & 1.07 & 1.15 & 1.24 \\
\hline
\end{longtable}

\item \textbf{Documentation match to life-cycle needs}:This parameter is evaluated in terms of the suitability of the project's documentation to its life-cycle needs. In our case, every need
of the product life-cycle is already present in the documentation,
so the DOCU cost driver is set to nominal.

\begin{longtable}{| m{1,8 cm}| m{1,8 cm} | m{1,8 cm} | m{1,8 cm} | m{1,8 cm} | m{1,8 cm} | m{1,8 cm}| }
\hline
\multicolumn{7}{c}{DOCU Cost Drivers}\\
\hline
\hline
DOCU Descriptors & Many life-cycle needs uncovered & Some life-cycle needs uncovered & Right-sized to life-cycle needs & Excessive for life-cycle needs & Very excessive for life-cycle needs & \\
\hline
Rating level & very low & low & nominal & high & very high & extra high \\
\hline
Effort multipliers & 0.81 & 0.91 & 1.00 & 1.11 & 1.23 & n/a \\
\hline
\end{longtable}

\item \textbf{Execution time constraint}:This parameter is a measure of the CPU usage imposed upon a software system. The rating is expressed in terms of the percentage of available execution time expected to be used by the system or subsystem consuming the execution time resource. In our case,since the software is quite complex,our our expectancy is that its CPU usage will be high.

\begin{longtable}{| m{1,8 cm}| m{1,8 cm} | m{1,8 cm} | m{1,8 cm} | m{1,8 cm} | m{1,8 cm} | m{1,8 cm}| }
\hline
\multicolumn{7}{c}{TIME Cost Drivers}\\
\hline
\hline
TIME Descriptors &  &  & \begin{equation*}
{\le 50\%} 
\end{equation*} use of available execution time & 70\% use of available execution time & 85\% use of available execution time & 95\% use of available execution time \\
\hline
Rating level & very low & low & nominal & high & very high & extra high \\
\hline
Effort multipliers & n/a & n/a & 1.00 & 1.11 & 1.29 & 1.63 \\
\hline
\end{longtable}

\item \textbf{Storage Constraint}: This parameter represents the degree of main storage constraint imposed on a software system or subsystem.As the modern disk drives can easily contain several terabytes of storage, this value is set to nominal.

\begin{longtable}{| m{1,8 cm}| m{1,8 cm} | m{1,8 cm} | m{1,8 cm} | m{1,8 cm} | m{1,8 cm} | m{1,8 cm}| }
\hline
\multicolumn{7}{c}{STOR Cost Drivers}\\
\hline
\hline
STOR Descriptors &  &  & \begin{equation*}
{\le 50\%} 
\end{equation*} use of available storage& 70\% use of available storage & 85\% use of available storage & 95\% use of available storage \\
\hline
Rating level & very low & low & nominal & high & very high & extra high \\
\hline
Effort multipliers & n/a & n/a & 1.00 & 1.05 & 1.17 & 1.46 \\
\hline
\end{longtable}

\item \textbf{Platform Volatility}:"Platform" is used here to mean the complex of hardware and software (OS, DBMS, etc.) the software product calls on to perform its tasks. In our case we don’t expect our fundamental platforms to change very often. However, the client applications may require at least a major release once every six months to be aligned with the development cycle of the main mobile operating systems. For this reason, this parameter is set to nominal.

\begin{longtable}{| m{1,8 cm}| m{1,8 cm} | m{1,8 cm} | m{1,8 cm} | m{1,8 cm} | m{1,8 cm} | m{1,8 cm}| }
\hline
\multicolumn{7}{c}{PVOL Cost Drivers}\\
\hline
\hline
PVOL Descriptors &  & Major change every 12 mo;minor change every 1 & Major change every 6mo;minor change every 2wk & Major change every 2mo;minor change every 1wk & Major change every 2wk;minor change every 2 days & \\
\hline
Rating level & very low & low & nominal & high & very high & extra high \\
\hline
Effort multipliers & n/a & 0.87 & 1.00 & 1.15 & 1.30 & n/a \\
\hline
\end{longtable}

\item\textbf{Analyst Capability}:Analysts are personnel that work on requirements, high level design and detailed design. The major attributes that should be considered in this rating are Analysis and Design ability, efficiency and thoroughness, and the ability to communicate and cooperate.We think the analysis of the problem has been conducted in very precise and complete way with respect to a potential real world
implementation. For this reason, this parameter is set to high.

\begin{longtable}{| m{1,8 cm}| m{1,8 cm} | m{1,8 cm} | m{1,8 cm} | m{1,8 cm} | m{1,8 cm} | m{1,8 cm}| }
\hline
\multicolumn{7}{c}{ACAP Cost Drivers}\\
\hline
\hline
ACAP Descriptors & 15th percentile  & 35th percentile & 55th percentile & 75th percentile & 90th percentile & \\
\hline
Rating level & very low & low & nominal & high & very high & extra high \\
\hline
Effort multipliers & 1.42 & 1.19 & 1.00 & 0.85 & 0.71 & n/a \\
\hline
\end{longtable}

\item \textbf{Programmer Capability}:Evaluation should be based on the capability of the programmers as a team rather than as individuals. Major factors which should be considered in the rating are ability, efficiency and thoroughness, and the ability to communicate and cooperate. The experience of the programmer should not be considered here; it is rated with AEXP.We have not implemented the project, so this parameter is just an estimation; however we think our programming ability are good,but the most important thing is that we know each other and the level of communication and cooperation is optimal, for this reason we set this parameter to high.

\begin{longtable}{| m{1,8 cm}| m{1,8 cm} | m{1,8 cm} | m{1,8 cm} | m{1,8 cm} | m{1,8 cm} | m{1,8 cm}| }
\hline
\multicolumn{7}{c}{PCAP Cost Drivers}\\
\hline
\hline
PCAP Descriptors & 15th percentile  & 35th percentile & 55th percentile & 75th percentile & 90th percentile & \\
\hline
Rating level & very low & low & nominal & high & very high & extra high \\
\hline
Effort multipliers & 1.34 & 1.15 & 1.00 & 0.88 & 0.76 & n/a \\
\hline
\end{longtable}

\item\textbf{Application Experience}:This rating is dependent on the level of applications experience of the project team developing the software system or subsystem. We have some experience in the development of Java applications,in particular we made just one project of considerable size,for this reason and for the fact that we never had experience with Java EE we set this parameter to low.

\begin{longtable}{| m{1,8 cm}| m{1,8 cm} | m{1,8 cm} | m{1,8 cm} | m{1,8 cm} | m{1,8 cm} | m{1,8 cm}| }
\hline
\multicolumn{7}{c}{APEX Cost Drivers}\\
\hline
\hline
APEX Descriptors & \begin{equation*}
{\le 2}
\end{equation*} months  & 6 months & 1 year & 3 years & 6 years & \\
\hline
Rating level & very low & low & nominal & high & very high & extra high \\
\hline
Effort multipliers & 1.22 & 1.10 & 1.00 & 0.88 & 0.81 & n/a \\
\hline
\end{longtable}

\item \textbf{Platform Experience}:
As we said before we don’t have any experience with the Java EE platform, but
we have some previous experience with databases, user interfaces
and server side development. For this reason, we set
this parameter to nominal.

\begin{longtable}{| m{1,8 cm}| m{1,8 cm} | m{1,8 cm} | m{1,8 cm} | m{1,8 cm} | m{1,8 cm} | m{1,8 cm}| }
\hline
\multicolumn{7}{c}{PLEX Cost Drivers}\\
\hline
\hline
PLEX Descriptors & \begin{equation*}
{\le 2}
\end{equation*} months  & 6 months & 1 year & 3 years & 6 years & \\
\hline
Rating level & very low & low & nominal & high & very high & extra high \\
\hline
Effort multipliers & 1.19 & 1.09 & 1.00 & 0.91 & 0.85 & n/a \\
\hline
\end{longtable}

\item\textbf{Language and Tool Experience}:This parameter measure the level of programming language and software tool experience of the project team developing the software system or subsystem. We don’t have any experience with the Java EE platform, but we have some previous experience with databases, user interfaces and server side development. We also know the development environment, so we’re going to set this parameter to nominal.

\begin{longtable}{| m{1,8 cm}| m{1,8 cm} | m{1,8 cm} | m{1,8 cm} | m{1,8 cm} | m{1,8 cm} | m{1,8 cm}| }
\hline
\multicolumn{7}{c}{LTEX Cost Drivers}\\
\hline
\hline
LTEX Descriptors & \begin{equation*}
{\le 2}
\end{equation*} months  & 6 months & 1 year & 3 years & 6 years & \\
\hline
Rating level & very low & low & nominal & high & very high & extra high \\
\hline
Effort multipliers & 1.20 & 1.09 & 1.00 & 0.91 & 0.84 & n/a \\
\hline
\end{longtable}

\item \textbf{Personnel continuity}:The rating scale for PCON is in terms of the project's annual personnel turnover. In our case since we have very strict deadlines and since the time we can
spend on this project is limited we set this parameter to very low.

\begin{longtable}{| m{1,8 cm}| m{1,8 cm} | m{1,8 cm} | m{1,8 cm} | m{1,8 cm} | m{1,8 cm} | m{1,8 cm}| }
\hline
\multicolumn{7}{c}{PCON Cost Drivers}\\
\hline
\hline
PCON Descriptors & 48\% / year & 24\% / year & 12\% / year & 6\% / year & 3\% / year & \\
\hline
Rating level & very low & low & nominal & high & very high & extra high \\
\hline
Effort multipliers & 1.29 & 1.12 & 1.00 & 0.90 & 0.81 & n/a \\
\hline
\end{longtable}

\textbf{Usage of Software Tools}: Our application environment is quite complete and well integrated, so
we set this parameter as nominal.

\begin{longtable}{| m{1,8 cm}| m{1,8 cm} | m{1,8 cm} | m{1,8 cm} | m{1,8 cm} | m{1,8 cm} | m{1,8 cm}| }
\hline
\multicolumn{7}{c}{TOOL Cost Drivers}\\
\hline
\hline
TOOL Descriptors & edit, code, debug & simple, frontend, backend CASE, little integration & basic life-cycle tools, moderately integrated & strong, mature life-cycle tools, moderately integrated & strong, mature, proactive life-cycle tools, well integrated with processes, methods, reuse
& \\
\hline
Rating level & very low & low & nominal & high & very high & extra high \\
\hline
Effort multipliers & 1.17 & 1.09 & 1.00 & 0.90 & 0.78 & n/a \\
\hline
\end{longtable}

\textbf{Multisite development}:Even if we live in different cities, we have met a considerable number of times and we have also collaborated hugely through Internet services including social networks,Skype,email
and Whatsapp. For this reason, we  set this parameter to very high.

\begin{longtable}{| m{1,8 cm}| m{2,3 cm} | m{1,8 cm} | m{1,8 cm} | m{1,8 cm} | m{1,8 cm} | m{1,8 cm}| }
\hline
\multicolumn{7}{c}{SITE Cost Drivers}\\
\hline
\hline
SITE Collocation Descriptors & International & Multi-city and multi-company & Multi-city or multi-company & Same city or metro area & Same building or complex  & Fully collocated \\
\hline
SITE Communication Descriptors & Some phone,email & Individual phone,fax & Narrow band email & Wideband electronic communication &  Wideband electronic comm. and occasionally video conf. & Interactive multimedia  \\ 
Rating level & very low & low & nominal & high & very high & extra high \\
\hline
Effort multipliers & 1.22 & 1.09 & 1.00 & 0.93 & 0.86 & 0.80 \\
\hline
\end{longtable}

\item \textbf{Required development schedule}: This rating measures the schedule constraint imposed on the project team developing the software. The ratings are defined in terms of the percentage of schedule stretch-out or acceleration with respect to a nominal schedule for a project requiring a given amount of effort. Our efforts were well distributed over the available development time, the definition of all the required documentation took a consistent amount of time. For this reason, this parameter
is set to high.

\begin{longtable}{| m{1,8 cm}| m{2,3 cm} | m{1,8 cm} | m{1,8 cm} | m{1,8 cm} | m{1,8 cm} | m{1,8 cm}| }
\hline
\multicolumn{7}{c}{SCED Cost Drivers}\\
\hline
\hline
SCED Descriptors & 75\% of nominal & 85\% of nominal & 100\% of nominal & 130\% of nominal & 160\% of nominal  &  \\
\hline
Rating level & very low & low & nominal & high & very high & extra high \\
\hline
Effort multipliers & 1.43 & 1.14 & 1.00 & 1.00 & 1.00 & n/a \\
\hline
\end{longtable}

\end{itemize}

To summarize the total results are expressed in the following table:
\begin{longtable}{| m{9 cm}| m{3 cm} | m{3 cm} | }
\hline
\textbf{Cost Drivers} & \textbf{Factor} & \textbf{Value} \\
\hline
Required Software Reliability(RELY) & nominal & 1.00 \\
\hline
Database Size(DATA) & nominal & 1.10 \\
\hline
Product Complexity(CPLX) & high & 1.17 \\
\hline
Required Reusability(RUSE) & nominal & 1.00 \\
\hline
Documentation match to life-cycle needs(DOCU) & nominal & 1.00 \\
\hline
Execution Time Constraint(TIME) & high & 1.11 \\
\hline
Main Storage Constraint(STOR) & nominal & 1.00 \\
\hline
Platform Volatility(PVOL) & nominal & 1.00 \\
\hline
Analyst Capability(ACAP) & high & 0.85 \\
\hline
Programmer Capability(PCAP) & high & 0.88 \\
\hline
Application Experience(APEX) & low & 1.10 \\
\hline
Platform Experience(PLEX) & nominal & 1.00 \\
\hline
Language and Tool Experience(LTEX) & nominal & 1.00 \\
\hline
Personnel Continuity(PCON) & very low & 1.29 \\
\hline
Usage of Software Tools(TOOL) & nominal & 1.00 \\
\hline
Multisite development(SITE) & very high & 0.86 \\
\hline
Required Development Schedule(SCED) & high & 1.00 \\
\hline
\multicolumn{2}{l}{Total} & 1.3040\\
\hline
\end{longtable}