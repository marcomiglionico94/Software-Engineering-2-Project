\section{Cost and effort estimation: COCOMO II}
In this section we are going to use the COCOMO II approach to estimate the
cost and effort needed to develop PowerEnJoy.
\subsection{Scale Drivers}
In order to evaluate the values of the scale drivers, we refer to the following
official COCOMO II table:

\begin{longtable}{| m{1,8 cm}| m{1,8 cm} | m{1,8 cm} | m{1,8 cm} | m{1,8 cm} | m{2,2 cm} | m{1,8 cm}| }
\hline
\textbf{Scale Factors} & \textbf{Very Low} & \textbf{Low} & \textbf{Nominal} & \textbf{High} & \textbf{Very High} & \textbf{Extra High}\\
\hline
PREC & thoroughly unprecedented & largely unprecedented & somewhat unprecedented & generally familiar & largely familiar & thoroughly familiar \\
\newline $SF_j$ & 6.20 & 4.96 & 3.72 & 2.48 & 1.24 & 0.00 \\
\hline
FLEX & rigorous & occasional relaxation & some relaxation & general conformity & some conformity & general goals \\
\newline $SF_j$ & 5.07 & 4.05 & 3.04 & 2.03 & 1.01 & 0.00 \\
\hline
RESL & little(20\%) & some(40\%) & often(60\%) & generally (75\%) & mostly(90\%) & full(100\%) \\
\newline $SF_j$ & 7.07 & 5.65 & 4.24 & 2.83 & 1.41 & 0.00 \\
\hline
TEAM & very difficult interactions & some difficult interactions & basically cooperative interactions & largely cooperative & highly cooperative & seamless interactions \\
\newline $SF_j$ & 5.48 & 4.38 & 3.29 & 2.19 & 1.10 & 0.00\\
\hline
PMAT & Level 1 Lower & Level 1 Upper & Level 2 & Level 3 & Level 4 & Level 5 \\
\newline $SF_j$ & 7.80 & 6.24 & 4.68 & 3.12 & 1.56 & 0.00\\
\hline
\end{longtable}

A brief description for each scale driver:
\begin{itemize}
\item\textbf{Precedentedness}:it reflects the previous experience of our team in the development of similar project. In our case, since we don't have this kind of experience the value will be low. 

\item\textbf{Development Flexibility}:it reflects the degree of flexibility in the development process with respect to the external specification and requirements.We set it to nominal because we have to follow a prescribed process, but we had a certain degree of flexibility in the definition of the requirements and in the design process.

\item\textbf{Architecture/Risk resolution}: it reflects the level of awareness and reactiveness with
respect to risks. The risk analysis we performed is quite extensive, so we have a clear definition of budget and schedule,for this reason the value will be set to very high.

\item\textbf{Team cohesion}: it reflects how well the development team know each other and work together. We set it to very high, since the cohesion among the three of us is optimal .

\item\textbf{Process maturity}: it reflects the process maturity of the organization. We set it to 3 because the  process comes very close to achieving the specific objectives such as quality, cost, and schedule. Besides processes, standards, procedures and tools were already defined
at the organizational level.
\end{itemize}

The estimated scale drivers for our project, together with the formula to
compute the exponent E are the following:
\begin{longtable}{| m{6 cm}| m{6 cm} | m{2 cm} |}
\hline
\textbf{Scale Drivers} & \textbf{Factor} & \textbf{Value} \\
\hline
Precedentedness(PREC) & Low & 4.96 \\
\hline
Development flexibility(FLEX) & Nominal & 3.04 \\
\hline
Risk Resolution(RESL) & Very High & 1.41 \\
\hline
Team Cohesion (TEAM) & Very High & 1.10 \\
\hline
Process Maturity (PMAT) & Level 3 & 3.12 \\
\hline
\textbf{Total} & & 13.63\\
\hline
\textbf{Total} & E=0.91 + 0.01 x $\sum_{1<=j<=5} SF_j$ & 1.0463 \\
\hline
\end{longtable}
