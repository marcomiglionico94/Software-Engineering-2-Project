A function point is a "unit of measurement" used to compute a functional size measurement (FSM) of software.
In the following we will use it to give an estimation of the size and complexity of our project.

This estimation is based on the usage of figures obtained through statistical analysis of real projects, which have been properly normalized and condensed in the following tables:

\begin{fptable}{For Internal Logic Files and External Logic Files}{Data Elements}
\fpvalues{Record Elements}{1-19}{20-50}{51+}
1 & Low & Low & Avg\\
2-5 & Low & Avg & High\\
6+ & Avg & High & High\\
\end{fptable}

\begin{fptable}{For External Output and External Inquiry}{Data Elements}
\fpvalues{File Types}{1-5}{6-19}{20+}
0-1 & Low & Low & Avg\\
2-3 & Low & Avg & High\\
4+ & Avg & High & High\\
\end{fptable}

\begin{fptable}{For External Input}{Data Elements}
\fpvalues{File Types}{1-4}{5-15}{16+}
0-1 & Low & Low & Avg\\
2-3 & Low & Avg & High\\
4+ & Avg & High & High\\
\end{fptable}

\begin{fptable}{UFP Complexity Weights}{Complexity Weight}
\fpvalues{Function Type}{Low}{Average}{High}
Internal Logic Files & 7 & 10 & 15\\
External Logic Files & 5 & 7 & 10\\
External Inputs & 3 & 4 & 6\\
External Outputs & 4 & 5 & 7\\
External Inquiries & 3 & 4 & 6\\
\end{fptable}