\section{Tier interactions}
\subsection{Integration test case SI1}

\begin{tabular}{l p{0.7\textwidth}}
    \hline
    \textbf{Test Case Identifier} & SI1T1\\
    \hline
    \textbf{Test Item(s)} & Business tier $\rightarrow$ Data Tier\\
    \hline
    \textbf{Input Specification} & Most typical calls (both correct and intentionally invalid ones) to the methods of the JPA Entities, which are mapped with tables in the Data tier.\\
    \hline
    \textbf{Output Specification} & Check if the Data tier does the correct queries on a test database. Also, it must react in the right way both if the requests are made correctly or wrongly and if they come from an unauthorized user.\\
    \hline
    \textbf{Environmental Needs} & Complete implementation of the Java Entity Beans; Java Persistence API; a test database; a driver that interacts with the Java Entity Beans. \\
    \hline
    \textbf{Test Description} & For each input, the response of the data tier will be compared with the expected output of the queries.\\
    \hline
    \textbf{Testing Method} & Automated with JUnit.\\
    \hline
\end{tabular}

\vspace{2em}

\subsection{Integration test case SI2}
\label{sec:performance-business}

\begin{tabular}{l p{0.7\textwidth}}
    \hline
    \textbf{Test Case Identifier} & SI2T1\\
    \hline
    \textbf{Test Item(s)} & Mobile application $\rightarrow$ Business Tier\\
    \hline
    \textbf{Input Specification} & Typical REST API calls (both correct and intentionally invalid ones) from the presentation tier to the business tier.\\
    \hline
    \textbf{Output Specification} & Check if the business tier responds accordingly to the API specification. Also, it must react in the right way both if the requests are made correctly or wrongly and if they come from an unauthorized user.\\
    \hline
    \textbf{Environmental Needs} & Complete implementation of the Business tier; a driver that simulates a mobile client through the REST API calls. \\
    \hline
    \textbf{Test Description} & For each API call of the clients, the response of the business tier will be compared with the expected output. The driver used for this test is a REST API client implemented in Java.\\
    \hline
    \textbf{Testing Method} & Automated with JUnit.\\
    \hline
\end{tabular}

\vspace{2em}

\noindent\begin{tabular}{l p{0.7\textwidth}}
    \hline
    \textbf{Test Case Identifier} & SI2T2\\
    \hline
    \textbf{Test Item(s)} & Mobile application $\rightarrow$ Business Tier\\
    \hline
    \textbf{Input Specification} & Multiple concurrent (typical and correct) requests to the REST API of the business tier.\\
    \hline
    \textbf{Output Specification} & Check if the business tier answers the requests in a reasonable amount time with respect to the applied load. \\
    \hline
    \textbf{Environmental Needs} & Complete implementation of the Business tier; GlassFish Server; Apache JMeter.\\
    \hline
    \textbf{Test Description} & This test case evaluates whether the business tier satisfy the performance requirements stated in the RASD (section 3.3, \emph{Performance requirements}).\\
    \hline
    \textbf{Testing Method} & Automated with Apache JMeter. \\
    \hline
\end{tabular}