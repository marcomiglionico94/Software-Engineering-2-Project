In this chapter the integration strategy will be described. In the section 2.1
the prerequisites for the tests will be presented. The section 2.2 is dedicated to
the required subsystems to be integrated in the system to execute some tests.
Finally, in the sections 2.3 and 2.4 the strategy used to test the integrations will
be discussed, paying attention to the order.
\section{Entry Criteria}
This section describes the prerequisites that need to be met before integration
testing can start and produce meaningful results.
All the classes and methods must pass thorough \textbf{unit tests}, which should
reasonably discover major issues in the structure of the classes or in the implementation
of the algorithms. Unit tests should have a minimum coverage
of 90\% of the lines of code and should be run automatically at each build
using JUnit. Unit testing is not in the scope of this document and will not
be specified in further detail.

Moreover, the \textbf{documentation} of all classes and functions, written using
JavaDoc, has to be complete and up-to-date, in order to be used as a reference
for integration testing development. In particular, the public interfaces of
each class and module should be well specified. Where necessary, a formal
specification language can be used.
The following \textbf{documents} must have been fully written before integration testing can
begin:

\begin{itemize}
\item \textbf{Requirement Analysis and Specification Document of PowerEnJoy};
\item \textbf{Design Document of PowerEnJoy};
\item \textbf{Integration Testing Plan Document of PowerEnJoy} (this document).
\end{itemize}

This a required phase to have a complete picture of the interactions between the different components of the system and of the functionalities they offer.

Finally, the integration testing phase can start also if some components don't have the minimum completion percentage necessary to consider it for integration (90\%), this is to reflect their order of integration and to take into account the required time to fully perform integration testing.

