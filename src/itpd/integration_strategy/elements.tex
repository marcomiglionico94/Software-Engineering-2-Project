\section{Elements to be Integrated}
In the following paragraph we are going to provide a list of all the components that need to be integrated together.

The integration process of our software is performed on two levels.
\begin{enumerate}
\item \textbf{Low Level:} integration of the different subcomponents (classes, Java Beans) inside the
same subsystem;
\item \textbf{High Level:} integration of different subsystems.
\end{enumerate}

The first step needs to be performed only for the components which contains
the pieces of software that we are going to develop, namely the business
tier, the mobile application in the client tier.
In particular the main subcomponents  that we will integrate are:
\begin{itemize}
\item \textbf{User Manager};
\item \textbf{Email Sender};
\item \textbf{Operator Safe Area Manager};
\item \textbf{User Safe Area Manager};
\item \textbf{Ride Manager};
\item \textbf{Car Manager};
\item \textbf{Fee Manager};
\item \textbf{Payment Manager}
\end{itemize}


The second step need to be performed on the three major high-level components that we outlined in the Design Document,that correspond to the tiers of the system, which – from now on – will
be referred to as subsystems:
\begin{itemize}
\item \textbf{Client tier:} The client tier consists of our mobile application,the operator terminal and the on-board tablet.
\item\textbf{Business tier:} This subsystem implements all the application logic and
communicates with the front-ends.
\item \textbf{Database tier:} This is the DBMS,it is not part of the software to be developed,
but has to be integrated.
\end{itemize}



