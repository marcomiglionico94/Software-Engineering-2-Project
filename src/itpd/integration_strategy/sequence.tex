\section{Sequence of Component/Function Integration}
\subsection{Software Integration Sequence}
In this section we are going to describe the order of integration (and integration
testing) of the various components of PowerEnJoy.
The components are tested starting from the most independent to the less
one.This because when less independent components are tested, the components which
they rely on have already been integrated. The components are integrated
within their classes in order to create an integrated subsystem which is ready
for subsystem integration.
As a notation, an arrow going from component C1 to component C2 means that C1 is necessary for C2 to function and so it must have already been implemented.

\subsection{Subsystem Integration Sequence}
\begin{tabular}{|c|c|c|c|}
\hline
\textbf{N.} & \textbf{Subsystem} & \textbf{Component} & \textbf{Integrates with}\\
\hline
\rule[-1cm]{0mm}{1cm}
I1 & Database,Business & (JEB)User & DBMS\\
\end{tabular}


