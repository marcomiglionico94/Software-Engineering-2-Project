\section{Sequence of Component/Function Integration}
\subsection{Software Integration Sequence}
In this section we are going to describe the order of integration (and integration
testing) of the various components of PowerEnJoy.

The components are tested starting from the most independent to the less
one. This because when less independent components are tested, the components which
they rely on have already been integrated. The components are integrated
within their classes in order to create an integrated subsystem which is ready
for subsystem integration.

As a notation, an arrow going from component C1 to component C2 means that C1 is necessary for C2 to function and so it must have already been implemented.
The integration sequence of the components is described in the following figure and table:

\includepdf{integration_strategy/Diagrams/IntegrationDiagram.png}

\begin{longtable}{| m{1cm}| m{3,5cm} | m{4cm} | m{8,7em} | }
\hline
\textbf{N.} & \textbf{Subsystem} & \textbf{Component} & \textbf{Integrates with}\\
\hline
I1 & Database,Business & (JEB)User & DBMS\\
\hline
I2 & Database,Business & (JEB)Ride & DBMS\\
\hline
I3 & Database,Business & (JEB)Car & DBMS\\
\hline
I4 & Database,Business & (JEB)Operator & DBMS\\
\hline
I5 & Database,Business & (JEB)Safe Area & DBMS\\
\hline
I6 & Database,Business & (JEB)Power Grid Station & DBMS\\
\hline
I7 & Business & (SB)Ride Manager & Car Manager \newline Fee Manager \newline Ride \newline Car\\
\hline
I8 & Business & (SB)Car Manager & Car \newline Ride Manager \\
\hline
I9 & Business & (SB)User Manager & User \newline User Safe Area Manager \newline Email Sender\\
\hline
I10 & Business & (SB)User Safe Area Manager & Safe Area\newline User \newline Power Grid Station\\
\hline
I11 & Business & (SB)Operator Safe Area Manager & Safe Area\newline Operator\newline Power Grid Station\newline User Safe Area Manager\\
\hline
I12 & Business & (SB)Fee Manager & Ride\\
\hline
I13 & Business & (SB)Payment Manager & User\\
\hline
I14 & Business & (EJB Container)Safe Area Container & Operator Safe Area Manager \newline User Safe Area Manager\\
\hline
I15 & Business & (EJB Container)User Manager Container & Email Sender \newline User Manager\\
\hline
I16 & Business & (EJB Container)Ride Manager Container & Ride Manager\\
\hline
I17 & Business & (EJB Container)Car Manager Container & Car Manager\\
\hline
I18 & Business & (EJB Container)Fee Manager Container & Fee Manager\\
\hline
I19 & Business & (EJB Container)Payment Manager Container & Payment Manager\\
\hline
I20 & Business & Controller & Safe Area Container \newline User Manager Container \newline Ride Manager Container \newline Car Manager Container \newline Car Manager Container \newline Fee Manager Container \newline Payment Manager Container\\
\hline
\end{longtable}

\subsection{Subsystem Integration Sequence}
A choice was made to proceed with the integration process from the database tier to the business tier and finally to the client tier. The reason to do so is that in order to have a functioning client you need to have a working business tier. The business tier, instead, can be tested without any client. 
The integration sequence of the subsystems is described in the following table and figure 2.1:
\begin{figure}
	\centering
	\includegraphics[scale=0.4]{integration_strategy/Diagrams/IntegrationSubsystem.png}
	\caption{Order of integration of the subsystems.}
	\label{fig:subsystems}
		
\begin{longtable}{| m{1cm}| m{4cm}| m{4cm} | }
\hline
\textbf{N.} & \textbf{Subsystem} & \textbf{Integrates with}\\
\hline
SI1 & Business Tier & Database Tier\\
\hline
SI2 & Mobile Application & Business Tier\\
\hline
SI3 & Power EnJoy Operator Terminal & Business Tier\\
\hline
SI4 & On-Board Tablet & Business Tier\\
\hline
SI5 & External System For Payment & Business Tier\\
\hline
SI6 & External System For Driving License Validation & Business Tier\\
\hline
SI7 & Handy Car Board & Business Tier\\
\hline
\end{longtable}


\end{figure}

