\section{High level components and their interaction}
Now we will present the main components of the system:
\begin{itemize}
\item \textbf{Mobile application:} This presentation layer consists in the mobile client. It communicate directly with the application server
\item \textbf{Application Server:} This layer contains all the application logic of the system. All the policies, the algorithms and the computation are performed here. This layer offers a service-oriented interface.
\item \textbf{Database:} The data layer is responsible for the data storage and retrieval. It does not implement any application logic. This layer must guarantee ACID properties. 
\end{itemize}

\begin{figure}
The system is structured in three layers as we can see in picture 2.1:
\begin{center}
\includegraphics[scale=0.6]{architectural_design/Architecture_Diagrams/Layers.png}
\caption{Layers of the system \label{fig:Layers}}
\end{center}
This design choice makes it possible to deploy the application server and the database on different tiers. It also improves scalability and fault tolerance.
\\
The figure 2.2 instead,show the three tiers from a very high level point of view:
\begin{center}
\includegraphics[scale=0.6]{architectural_design/Architecture_Diagrams/Tier.png}
\caption{Tiers of the system \label{fig:Layers}}
\end{center}
\end{figure}
\begin{figure}
The interactions between the main components are shown in the figure 2.3:
\begin{center}
\includegraphics[scale=0.6]{architectural_design/Architecture_Diagrams/Tier2.png}
\caption{High level components of the system \label{fig:Layers}}
\end{center}
\end{figure}
\begin{figure}
A more detailed description of the three different tiers is shown in picture 2.4:
\begin{center}
\includegraphics[scale=0.4]{architectural_design/Architecture_Diagrams/Tier3.png}
\caption{Detailed description of the tiers,detailed with Java EE components \label{fig:Layers}}
\end{center}
\end{figure}






