\section{Overview}
In this section we will present the high level components of the system and their interaction:

\begin{description}
\item[Mobile application:] This presentation layer consists in the mobile client. It communicates directly with the application server.
\item[Application Server:] This layer contains all the application logic of the system. All the policies, the algorithms and the computation are performed here. This layer offers a service-oriented interface.
\item[Database:] The data layer is responsible for the data storage and retrieval. It does not implement any application logic. This layer must guarantee ACID properties. 
\item[HandyCar board:] A board to which all the actuators and sensors of the car are connected.
It communicates with the application server, abstracting all the low-level details of the cars.
\item[On-Board tablet:] A tablet that is on board of the vehicles. It used for the communication between the system and the driver, so the tablets communicates with the application server.
\item[PowerEnJoy Operator Program:] A program that the operators of PowerEnJoy can use to interact with the system, for example modifying the safe areas. It communicates with the application server.
\item[External Systems:] There are three external systems: 
	\begin{itemize}
	\end{itemize}
\end{description}

\begin{figure}
	\centering
	\includegraphics[scale=0.6]{architectural_design/Architecture_Diagrams/Layers.png}
	\caption{Layers of the system.}
	\label{fig:layers}
\end{figure}

The system is structured in three layers as we can see in picture \ref{fig:layers}.

This design choice makes it possible to deploy the application server and the database on different tiers. It also improves scalability and fault tolerance.

\begin{figure}
	\centering
	\includegraphics[scale=0.6]{architectural_design/Architecture_Diagrams/Tier.png}
	\caption{Tiers of the system.}
	\label{fig:tiers}
\end{figure}

The figure \ref{fig:tiers}, instead, shows the three tiers from a very high level point of view: Client,Server and Data.

\begin{figure}
	\centering
	\includegraphics[width=\textwidth,height=\dimexpr\textheight-4\baselineskip-\abovecaptionskip-\belowcaptionskip\relax,keepaspectratio]{architectural_design/Architecture_Diagrams/Tier2.png}
	\caption{High level components of the system.}
	\label{fig:high_components}
\end{figure}

The interactions between the main components are shown in the figure \ref{fig:high_components}.

\begin{figure}
    \vspace*{-2cm}
    \makebox[\linewidth]{
        \includegraphics[width=1.3\linewidth]{architectural_design/Architecture_Diagrams/Tier3.png}
    }
    \caption{Description of the tiers, detailed with Java EE components.}
	\label{fig:tiers_description}
\end{figure}

A more detailed description of the three different tiers is shown in picture \ref{fig:tiers_description}.


