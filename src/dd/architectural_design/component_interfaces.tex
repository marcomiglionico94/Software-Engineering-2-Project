\section{Component Interfaces}

\subsection{Logic layer to data storage layer}
The logic layer communicates with the data storage layer via the Java Persistence API (JPA) over standard network protocols.
In this way, the two layers can be deployed both in different tiers or in the same one.

The JPA specification uses an object/relational mapping approach to bridge the gap between an object-oriented model and a relational database in order to focus more on the object model rather than on the actual SQL queries used to access data stores.

\subsection{Logic layer to presentation layer}

The logic layer and the presentation layer communicates with each other over HTTPS protocol %TODO

\subsubsection{UserManager}
\begin{description}
	\item[Register] This function add a new user in the system with the provided data if these are correct.
	After this, an email with a token is sent to the user email address in order to confirm the latter.
    \item[Login] This function allows any registered user to log into the system using his own username and password.
    If these credentials are correct, the function returns a token to be used in the future requests to identify the user, otherwise it returns an error.
    \item[Comfirm Email] This function validates the email address that was inserted by a registered user using the token sent to that email address after the registration.
    \item[Delete User] This function validates the email address that was inserted by a registered user using the token sent to that email address after the registration.
\end{description}
