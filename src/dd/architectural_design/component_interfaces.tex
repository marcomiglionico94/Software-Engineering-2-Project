\section{Component Interfaces}

\subsection{Logic layer to data storage layer}
The logic layer communicates with the data storage layer via the Java Persistence API (JPA) over standard network protocols.
In this way, the two layers can be deployed both in different tiers or in the same one.

The JPA specification uses an object/relational mapping approach to bridge the gap between an object-oriented model and a relational database in order to focus more on the object model rather than on the actual SQL queries used to access data stores.

\subsection{Logic layer to presentation layer}

The application server communicates with the other elements of the system through a RESTful interface over the HTTPS protocol.
The RESTful interface is implemented using JAX-RS and uses JSON as the language data format.

\subsubsection{User Manager}
All the following functions can be called by the mobile app of the user:
\begin{itemize}
	\item \textbf{void register(String username, String password, String email):} Add a new user in the system with the provided data if these are correct.
	After this, an email with a token is sent to the user email address in order to confirm the latter.
    \item \textbf{Token login(String username, String password):} Allows any registered user to log into the system using his own username and password.
    If these credentials are correct, the function returns a token to be used in the future requests to identify the user, otherwise it returns an error.
    \item \textbf{void confirmEmail(String username, Token emailToken):} Validates the email address that was inserted by a registered user using the token sent to that email address after the registration.
    \item \textbf{void deleteUser(String username, String password):} Allows users to delete the user account and information, except for his essential information and rides because they can be requested from authorities. As parameters it takes the username of the user and his password.
	\item \textbf{void editProfile(String fieldName, String newValue):} Allows users to edit their information. If they user wants to change its email, he/she has to confirm it as for the registration.
	\item \textbf{void userBanning(String username):} Blocks the user's account.% if he/she has some pending bills until it will be paid.
	This function is available only for PowerEnJoy operators.
	%\item \textbf{Driving license validation:} Check the validity of the driver license provided by the user through an external system.
	%\item \textbf{Payment information verification:} Check the correctness of the payment information provided by the user through an external system.

\end{itemize}

\subsubsection{Ride Manager}
\begin{itemize}
	\item \textbf{Ride getRide(String id):} Returns the ride info that corresponds to a given ID. The function will return an error if a user wants to retrieve the info about a ride that is not assigned to him/her.
	The PowerEnJoy operators can get all the rides, without restrictions.
	The info about a ride contain: the plate of the car, the username of the user that reserved the car, unlock time, ignition time, end time, the fee with variations and the fee variations and the maximum number of passengers.
%	\item \textbf{Add ride:} Registers a new ride with the start time, the unlock time, %TODO the ignition time, 
%the plate of the car and the username of the user.
	\item \textbf{List \textless Ride\textgreater{} getUserRides(String username):} Returns the rides info that corresponds to a given users. The function will return an error if a user wants to retrieve the info about another user.
	The PowerEnJoy operators can get all the rides, without restrictions.
\end{itemize}

\subsubsection{Car Manager}
\begin{itemize}
	\item \textbf{Car getCar(String plate):} Returns the car info that corresponds to a given plate.
	The info about a car contain: the battery level, the state, if the engine is on or off, the number of seats, the model and the manufacturer.
	\item \textbf{void reserveCar(String plate):} The mobile app calls this function to allow the user to request a car. This function returns an error if the request is not valid.
	\item \textbf{void unlockCar(String plate):} The mobile app calls this function to allow the user to unlock a car that he reserved. This function returns an error if the request is not valid.
%	\item \textbf{Update position:} Updates the position of the car. This function is called periodically to monitor the movements of the cars.
	\item \textbf{void endRide():} The HandyCar Board use this function to report to the system that the user ended a ride. At this point, the application server saves the end time, the fee variation that were applied and the maximum number of passengers. Every HandyCar Board has its own ID, so the system can recognize from which car the request is sent.
\end{itemize}

\subsubsection{Fee Manager}
All this functions can be called from the user of the ride or from PowerEnJoy operators.
\begin{itemize}
	\item \textbf{int getFeeVariation(String ID):} Calculates a fee variation for a given ride ID. 
	\item \textbf{int getFeeWithoutVariation(String ID):} Calculates a fee for a given ride, without including the variations of bonuses and penalty.
	\item \textbf{int getFeeWithVariation(String ID):} Calculates a fee for a given ride, including the variations of bonuses and penalty.
\end{itemize}
