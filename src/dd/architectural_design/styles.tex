\section{Selected architectural styles and patterns}
%Please explain which styles/patterns you used, why, and how
We used the following architectural styles and patterns:

\begin{description}
\item[Client-Server:]
The client–server model is used between these communications:
	\begin{itemize}
    \item the application server (client) queries the DB (server);
    \item the presentation layer (client) communicates with the application server (server);
    \item the application server (client) and the on-board tablet (server);
    \item the HandyCar Board and the application server (both act like client and server);
    \item the application server (client) and the payment validation external system (server);
    \item the application server (client) and the driving license validation external system (server).
	\end{itemize}

We choose this type of architecture to have a centralized control and because it is simple and well known by our development team.

\item[Service-oriented architecture:]
The SOA is used by the system for the communication between the application server and the presentation layer.

We used SOA in order to have an high level interaction between these two layers, by looking only at the component interfaces, that represent a black box for its consumers.

In this way, a service logically represents a business activity with a specified outcome, so you want to use it, you do not have to look inside its specific implementation.

\item[Thin client:]
The thin client is implemented in the mobile app and in the on-board tablet.
We used this paradigm in order to have light-weight apps, that can run in a large number of devices with a smooth user interaction.

Instead, all the application logic is on the application server, on which the client depends heavily on.
This can help to reduce the number of updates, because if you want to change something in the application logic, it is not necessary to release a new application of the client.

\end{description}
% \subsection{Model-View-Controller}
